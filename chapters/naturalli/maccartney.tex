\Section{maccartney-proof}{MacCartney's Proofs By Aligmnent}


\newcite{key:2007maccartney-natlog} approach inference for natural logic in
  the context of inferring whether a single relevant premise entails
  a query.
Their approach first generates an alignment between the premise
  and the query, and then classifies each aligned segment into one of
  the lexical relations described in \refchp{natlog}.
Inference reduces to projecting each of these relations
  according to the polarity function (\reftab{natlog-mono-add})
  and iteratively \textit{joining}  two projected relations together to 
  get the final entailment relation.
This join relation, denoted as $\bowtie$, is given in \reftab{natlog-jointable}.

To illustrate, we can consider MacCartney's example inference from
  \w{Stimpy is a cat} to \w{Stimpy is not a poodle}.
An alignment of the two statements would provide three lexical
  mutations:
    \mbox{$r_1\coloneqq\w{cat}\rightarrow\w{dog}$}, 
    \mbox{$r_2\coloneqq\cdot\rightarrow\w{not}$},
    and \mbox{$r_3\coloneqq\w{dog}\rightarrow\w{poodle}$}.
%The initial relation $r_0$ is axiomatically \equivalent.
Each of these are then projected with the projection function $\rho$,
  and are joined using the join relation:

\begin{equation*}
  r_0 \bowtie \rho(r_1) \bowtie \rho(r_2) \bowtie \rho(r_3),
\end{equation*}

\noindent where the initial relation $r_0$ is axiomatically \equivalent.
In MacCartney's work this style of proof is presented as a table.
The last column ($s_i$) is the relation between the premise and the
  $i^{th}$ step in the proof, and is constructed inductively as
  $s_i \coloneqq s_{i-1} \bowtie \rho(r_i)$:

\begin{center}
\begin{tabular}{rl|ccc}
  \multicolumn{2}{c|}{Mutation} & $r_i$ & $\rho(r_i)$ & $s_i$ \\
  \hline
  $r_1$ & \w{cat}$\rightarrow$\w{dog}    & \alternate & \alternate & \alternate \\
  $r_2$ & $\cdot\rightarrow$\w{not}      & \negate    & \negate    & \forward \\
  $r_3$ & \w{dog}$\rightarrow$\w{poodle} & \reverse   & \forward   & \forward \\
\end{tabular}
\end{center}

In our example, we would conclude that
  \w{Stimpy is a cat} \forward\ \w{Stimpy is not a poodle}
  since $s_3$ is \forward;
  therefore the inference is valid.
More details on natural logic can be found in \refchp{natlog}.
