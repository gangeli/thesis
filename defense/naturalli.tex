%%%%%%%%%%%%%%%%%%%%%%%%%%%%%%%%%%%%%%%%%%%%%%%%%%%%%%%%%%%%%%%%%%%%%%%%%%%%%%%
% MOTIVATION
%%%%%%%%%%%%%%%%%%%%%%%%%%%%%%%%%%%%%%%%%%%%%%%%%%%%%%%%%%%%%%%%%%%%%%%%%%%%%%%

%%%%%%%%%%%%%%%%%%% 
% COMMON SENSE REASONING TEASER
%%%%%%%%%%%%%%%%%%%
\begin{frame}{Reasoning About Common Sense Facts}
\begin{tabular}{cc}
  \true{Kittens play with yarn} & \false{Kittens play with computers} \\
  \vspace{0.25cm} \\
  \includegraphics[width=5cm]{../img/yarn-cat.png} & \pause \includegraphics[width=5cm]{../img/computer-cat-cropped.jpg}
\end{tabular}
\end{frame}


%%%%%%%%%%%%%%%%%%% 
% COMMON SENSE REASONING NLP
%%%%%%%%%%%%%%%%%%%
\begin{frame}{Common Sense Reasoning for NLP}
\begin{center}
\includegraphics[width=8cm]{../img/ambiguity.png}
\end{center}
\end{frame}
%\begin{center}
%  \w{The city refused the demonstrators a permit because they feared violence.} \\
%  \pause
%  \begin{tabular}{l}
%    \true{a city fears violence} \\
%    \false{demonstrators fear violence}
%  \end{tabular}
%\end{center}
%\pause
%
%\begin{center}
%  \w{I ate the cake with a cherry} \hspace{0.25cm} vs. \hspace{0.25cm} \w{I ate the cake with a fork} \\
%  \begin{tabular}{l}
%    \true{cakes come with cherries} \\
%    \false{cakes are eaten using cherries}
%  \end{tabular}
%\end{center}
%\pause
%
%\begin{center}
%  \w{Put a sarcastic comment in your talk. That's a great idea.} \\
%  \begin{tabular}{l}
%    \false{Sarcasm in your talk is a great idea}
%  \end{tabular}
%\end{center}
%
%\end{frame}

%%%%%%%%%%%%%%%%%%%% 
%% COMMON SENSE REASONING VISION
%%%%%%%%%%%%%%%%%%%%
%\begin{frame}{Common Sense Reasoning for Vision}
%\begin{tabular}{cc}
%  \false{Dogs drive cars} & \true{People drive cars} \\
%  \includegraphics[height=3cm]{../img/dog-driving.jpg} & \includegraphics[height=3cm]{../img/person-driving.jpg}  \\
%  \vspace{0.0cm} \\
%  \pause \false{Baseball is played underwater} & \true{Baseball is played on grass} \\
%  \includegraphics[height=3cm]{../img/baseball-underwater.jpg} & \includegraphics[height=3cm]{../img/baseball-grass.jpg} \\
%\end{tabular}
%\end{frame}

%%%%%%%%%%%%%%%%%%%% 
%% COMMON SENSE CURRENTLY
%%%%%%%%%%%%%%%%%%%%
%\begin{frame}{Prior Work on Common Sense Reasoning}
%\hh{Old School AI:} Nuanced reasoning; tiny coverage.
%\begin{itemize}
%  \item Default reasoning \cite{key:1980reiter-logic,key:1980mccarthy-circumscription}
%  \item Theorem proving (e.g., Datalog).
%\end{itemize}
%\vspace{0.5cm}
%\pause
%
%\hh{Textual Entailment:} Rich inference; small data.
%\begin{itemize}
%  \item RTE Challenges.
%  \item Episodic Logic \cite{key:2002schubert-commonsense}.
%\end{itemize}
%\vspace{0.5cm}
%\pause
%
%\hh{Information Extraction:} Shallow inference, large data.
%\begin{itemize}
%  \item OpenIE \cite{key:2007yates-textrunner}, NELL \cite{key:2010carlson-nell}.
%  \item \textit{Extraction} of facts from a large corpus; fuzzy lookup.
%%  \pause
%%  \item \textbf{Inference doesn't 10x the knowledge base size.}
%\end{itemize}
%\end{frame}

%%%%%%%%%%%%%%%%%%%% 
%% OpenIE
%%%%%%%%%%%%%%%%%%%%
%\begin{frame}{Open Information Extraction}
%\hh{Mismatch in Evaluation} \\
%\vspace{0.25cm}
%\begin{tabular}{ll}
%  \h{IE Metric} & ``How many facts did I extract'' \\
%  \h{Common Use Case} & ``Is this query in my knowledge base?''
%\end{tabular}
%\end{frame}

%%%%%%%%%%%%%%%%%%% 
% TEASER
%%%%%%%%%%%%%%%%%%%
\begin{frame}{Start with a large knowledge base}
\begin{center}
  \hspace{0.8cm}
  \includegraphics[height=2.2cm]{../img/database.png}
\end{center}
\end{frame}

\begin{frame}[noframenumbering]{Start with a large knowledge base}
\begin{center}
  \teaserManyPremises
\end{center}
\end{frame}

\begin{frame}[noframenumbering]{Infer new facts...}
\begin{center}
  \teaserBlindInferenceNaturalOrderBlind
\end{center}
\end{frame}

\begin{frame}[noframenumbering]{Infer new facts...}
\begin{center}
  \teaserBlindInferenceNaturalOrder
\end{center}
\pause
\begin{textblock*}{6.0cm}(6.5cm,5cm)
  \textcolor<1-1>{white}{$\uparrow$ Don't want to run inference \\ $~~$ over every fact!}
\end{textblock*}
\pause
\begin{textblock*}{6cm}(5.5cm,7.6cm)
  \textcolor<1-2>{white}{$\leftarrow$ Don't want to store all of these!}
\end{textblock*}
\end{frame}

\begin{frame}[noframenumbering]{Infer new facts...on demand from a query...}
\begin{center}
  \teaserBlindInference
\end{center}
\end{frame}

\begin{frame}[noframenumbering]{...Using text as the meaning representation...}
\begin{center}
  \teaserInference
\end{center}
\end{frame}

\begin{frame}[noframenumbering]{...Without aligning to any particular premise.}
\begin{center}
  \teaserFullDerivation
\end{center}
\end{frame}

\begin{frame}[noframenumbering]{Infer new facts...}
\begin{center}
  \teaserBlindInferenceNaturalOrder
\end{center}
\end{frame}

%%%%%%%%%%%%%%%%%%%%%%%%%%%%%%%%%%%%%%%%%%%%%%%%%%%%%%%%%%%%%%%%%%%%%%%%%%%%%%%
% NATURAL LOGIC
%%%%%%%%%%%%%%%%%%%%%%%%%%%%%%%%%%%%%%%%%%%%%%%%%%%%%%%%%%%%%%%%%%%%%%%%%%%%%%%

%%%%%%%%%%%%%%%%%%% 
% Why not FOL?
%%%%%%%%%%%%%%%%%%%
\begin{frame}{We're Doing Logical Inference}
\begin{center}
\w{The cat ate a mouse} \textbf{$\vDash \lnot$} \darkred{\textit{No carnivores eat animals}}
\end{center}
\pause
\vspace{2ex}

\hh{Recall:} Inference on every query: \textbf{speed is important!} \\
\pause
\vspace{2ex}
\hh{Recall:} Both premise and query are sentences. \\
\pause
\vspace{2ex}

\hh{Detour:} Let's talk about logic!
\end{frame}

%%%%%%%%%%%%%%%%%%%
% INTRACTABLE
%%%%%%%%%%%%%%%%%%%
\begin{frame}{First Order Logic is \textbf{Intractable}}
\begin{center}
  \includegraphics[height=2cm]{../img/overheating.jpg}
\end{center}

\hh{Theorem Provers}
\begin{itemize}
  \item Propositional logic is already NP-complete!
\end{itemize}
\vspace{1ex}
\pause

\hh{Markov Logic Networks}
\begin{itemize}
  \item Grounding 3,800 rules takes 7 hours (Alchemy).
  \pause
  \item Add Chris R\'{e} $+$ decades of DB research: 106 seconds.
  \item $\dots$ but still slow for open-domain inference.
\end{itemize}

\end{frame}

%%%%%%%%%%%%%%%%%%%
% HARD TO PARSE
%%%%%%%%%%%%%%%%%%%
\begin{frame}{First order logic is \textbf{an unnatural language}}
%\begin{center}
%  \includegraphics[height=3cm]{../img/math.jpg}
%\end{center}
%\vspace{1ex}

\begin{center}
$\exists x \left( \textrm{location}(x) \land \textrm{born\_in}(\textrm{Obama}, x) \right)$
\end{center}
\vspace{1ex}

\begin{center}
  \includegraphics[height=4cm]{../img/sempre.png}
\end{center}

\footnotetext<.->{\cite{key:2013berant-sempre}}
\end{frame}


%%%%%%%%%%%%%%%%%%%
% UNEXPRESSIVE
%%%%%%%%%%%%%%%%%%%
\begin{frame}{First Order Logic is \textbf{Inexpressive}}
\begin{center}
  \includegraphics[height=2cm]{../img/pokerface.png}
\end{center}

\hh{\textit{Some people think that Obama was born in Kenya.}}
\begin{itemize}
\item Second order logic: 
      $\exists x \exists P \left[ P = \textrm{born} \land \textbf{\textrm{think}(x, P)} \land P(\textrm{Obama}, \textrm{Kenya}) \right]$
\pause
\item But, can still infer: \w{Some people think that Obama is from Kenya.}
\end{itemize}
\vspace{1ex}
\pause


\hh{\textit{Most students who learned a foreign language learned it at a university.}}
\begin{itemize}
\item \w{Most} is not a first-order quantifier.
\item Scoping ambiguities everywhere!
\pause
\item But, can still infer: \w{Most students learned it at a school.}
\end{itemize}
\end{frame}


%%%%%%%%%%%%%%%%%%% 
% Natural Logic
%%%%%%%%%%%%%%%%%%%
\def\title{Natural Logic}
\begin{frame}{\title}
\begin{center}
  \includegraphics[height=5cm]{../img/aristotle.png}
\end{center}
\footnotetext<.->{\cite{key:1991valencia-natlog,key:2008maccartney-natlog,key:2014icard-natlog}}
\end{frame}

\begin{frame}{\title}
\begin{center}
\hh{Does a given mutation to a sentence preserve its truth?}
\end{center}
\vspace{1em}
\pause

\hh{Logic over natural language}
\begin{itemize}
\item \textit{Instantaneous} and \textit{perfect} semantic parsing!
\item Plays nice with lexical methods
\end{itemize}
\vspace{1ex}
\pause

\hh{Tractable}
\begin{itemize}
\item Polynomial time entailment checking \cite{key:2008maccartney-natlog}.
\end{itemize}
\vspace{1ex}
\pause

\hh{Expressive (for common inferences)}
\begin{itemize}
\item Second-order phenomena; \w{most}; quantifier scoping
\pause
\item No free lunch: shallow quantification; single-premise only
\end{itemize}
\end{frame}


%%%%%%%%%%%%%%%%%%% 
% NATLOG AS SYLLOGISMS
%%%%%%%%%%%%%%%%%%%


\def\title{Natural Logic as Syllogisms}
\begin{frame}[noframenumbering]{\title}
\begin{center}
  \hh{s/Natural Logic/Syllogistic Reasoning/g} \\
  \vspace{0.25cm}
  \begin{tabular}{lp{4cm}}
    &Some cat ate a mouse \\
    & \darkgray{\textit{(all mice are rodents)}} \\
    $\therefore$& \w{Some cat ate a \textbf{rodent}} \\
  \end{tabular}
\end{center}
\vspace{3ex}
\pause

\hh{Beyond syllogisms}
\begin{itemize}
\item General-purpose logic
\begin{itemize}
  \item Compositional grammar
  \item Arbitrary quantifiers
\end{itemize}
\item Model-theoretic soundness $+$ completeness proof \cite{key:2014icard-natlog}
\end{itemize}
\end{frame}


%%%%%%%%%%%%%%%%%%% 
% POLARITY
%%%%%%%%%%%%%%%%%%%
\def\catFelineVenn{
  \begin{tikzpicture}
    \def\vennA{(-0.1,-0.1) circle (0.2)}
    \def\vennB{(-0.0,-0.0) circle (0.5)}

    \draw \vennB node [below] {};
    \draw \vennA node [above] {};
    \begin{scope}
      \fill[fill=light] \vennB;
    \end{scope}
    \begin{scope}
      \fill[fill=dark] \vennA;
    \end{scope}
    
    \frameVenn
    \draw (0, -1.3) node {\w{cat} $\subseteq$ \w{feline}};
  \end{tikzpicture}
}

\def\felineAnimalVenn{
  \begin{tikzpicture}
    \def\vennA{(-0.1,-0.1) circle (0.2)}
    \def\vennB{(-0.0,-0.0) circle (0.5)}

    \draw \vennB node [below] {};
    \draw \vennA node [above] {};
    \begin{scope}
      \fill[fill=dark] \vennB;
    \end{scope}
    \begin{scope}
      \fill[fill=light] \vennA;
    \end{scope}
    
    \frameVenn
    \draw (0, -1.3) node {\w{feline} $\subseteq$ \w{animal}};
  \end{tikzpicture}
}

\def\header{
  \hh{Treat hypernymy as a \textit{partial order}.}
  \begin{center}
%    \catFelineVenn \felineAnimalVenn 
%    \hspace{0.5cm}
%    \raisebox{1cm}[0pt][0pt]{
%      \includegraphics[height=1.0cm]{../../img/rArrow.jpg}
%    }
%    \hspace{0.5cm}
    \scalebox{0.75}{\lattice}
  \end{center}
}
\def\blurb{
  \hh{\textit{Polarity} is the direction a lexical item can move in the ordering.} \\
}
\def\title{Natural Logic and Polarity}

\begin{frame}{\title}
  \header
  \pause
  \blurb
  \begin{center}
    \monoHeader
      \node[black]{animal};
      \node[black]{feline};
      \node[punktchain,color=blue]{cat};
      \node[black]{house cat};
    \end{tikzpicture}
  \end{center}
\end{frame}

\begin{frame}[noframenumbering]{\title}
  \header
  \blurb
  \begin{center}
    \monoUp{house cat}{cat}{feline}{animal}
  \end{center}
\end{frame}

\begin{frame}[noframenumbering]{\title}
  \header
  \blurb
  \begin{center}
    \monoUp{cat}{feline}{animal}{living thing}
  \end{center}
\end{frame}

\begin{frame}[noframenumbering]{\title}
  \header
  \blurb
  \begin{center}
    \monoUp{feline}{animal}{living thing}{thing}
  \end{center}
\end{frame}

\begin{frame}[noframenumbering]{\title}
  \header
  \blurb
  \begin{center}
    \monoDown{feline}{animal}{living thing}{thing}
  \end{center}
\end{frame}

\begin{frame}[noframenumbering]{\title}
  \header
  \blurb
  \begin{center}
    \monoDown{cat}{feline}{animal}{living thing}
  \end{center}
\end{frame}

\begin{frame}[noframenumbering]{\title}
  \header
  \blurb
  \begin{center}
    \monoDown{house cat}{cat}{feline}{animal}
  \end{center}
\end{frame}

%%%%%%%%%%%%%%%%%%% 
% NATLOG ANIMATION
%%%%%%%%%%%%%%%%%%%
\input example.tex

%%%%%%%%%%%%%%%%%%%%%%%%%%%%%%%%%%%%%%%%%%%%%%%%%%%%%%%%%%%%%%%%%%%%%%%%%%%%%%%
% INFERENCE
%%%%%%%%%%%%%%%%%%%%%%%%%%%%%%%%%%%%%%%%%%%%%%%%%%%%%%%%%%%%%%%%%%%%%%%%%%%%%%%


%%%%%%%%%%%%%%%%%%% 
% REMINDER
%%%%%%%%%%%%%%%%%%%
\begin{frame}[noframenumbering]{Infer new facts...}
\begin{center}
  \teaserBlindInferenceNaturalOrder
\end{center}
\end{frame}
\begin{frame}[noframenumbering]{Infer new facts...on demand from a query...}
\begin{center}
  \teaserBlindInference
\end{center}
\end{frame}

\begin{frame}[noframenumbering]{...Using text as the meaning representation...}
\begin{center}
  \teaserInference
\end{center}
\end{frame}

\begin{frame}[noframenumbering]{...Without aligning to any particular premise.}
\begin{center}
  \teaserFullDerivation
\end{center}
\end{frame}

%%%%%%%%%%%%%%%%%%% 
% INFERENCE IS SEARCH
%%%%%%%%%%%%%%%%%%%
\begin{frame}{Natural Logic Inference is Search}
\begin{center}
  \includegraphics[width=5cm]{../img/dijkstras-graph.pdf}
\end{center}
\begin{tabular}{ll}
  \hh{Nodes} & $($ \w{fact}, truth maintained $\in\{\textrm{true}, \textrm{false}\})$ \\
  & \\
  \pause
  \hh{Start Node} & $($ \w{query fact}, \true{true} $)$ \\
  \hh{End Nodes}  & \w{any known fact} \\
  & \\
  \pause
  \hh{Edges} & Mutations of the current fact \\
  \pause
  \hh{Edge Costs} & How ``wrong'' an inference step is (learned) \\
\end{tabular}
\end{frame}

%%%%%%%%%%%%%%%%%% 
% EXAMPLE SEARCH
%%%%%%%%%%%%%%%%%%
\input exampleSearch.tex

%%%%%%%%%%%%%%%%%%% 
% EDGE TEMPLATES
%%%%%%%%%%%%%%%%%%%
\begin{frame}{Edge Templates}
\begin{center}
  \begin{tabular}{p{0.4\textwidth}p{0.20\textwidth}}
    \multicolumn{1}{c}{\textbf{Template}} & \multicolumn{1}{c}{\textbf{Instance}} \\
    Hypernym & \w{animal} $\rightarrow$ \w{cat} \\
    Hyponym  & \w{cat} $\rightarrow$ \w{animal} \\
    Antonym  & \w{good} $\rightarrow$ \w{bad} \\
    Synonym  & \w{cat} $\rightarrow$ \w{true cat} \\
    & \\
    Add Word  & \w{cat} $\rightarrow$ \w{$\cdot$} \\
    Delete Word  & $\cdot$ $\rightarrow$ \w{cat} \\
    & \\
    Operator Weaken & \w{some} $\rightarrow$ \w{all} \\
    Operator Strengthen & \w{all} $\rightarrow$ \w{some} \\
    Operator Negate & \w{all} $\rightarrow$ \w{no} \\
    Operator Synonym & \w{all} $\rightarrow$ \w{every} \\
    & \\
    Nearest Neighbor  & \w{cat} $\rightarrow$ \w{dog} \\
  \end{tabular}
\end{center}
\end{frame}


%%%%%%%%%%%%%%%%%%% 
% SOFT LOGIC
%%%%%%%%%%%%%%%%%%%
\begin{frame}{``Soft'' Natural Logic}
\hh{Want to make likely (but not certain) inferences}.
\begin{itemize}
  \item Same motivation as Markov Logic, Probabilistic Soft Logic, etc.
  \pause
  \item Each \textit{edge template} has a cost $\theta \geq 0$.
\end{itemize}
\vspace{0.5cm}
\pause

\hh{Detail:} Variation among \textit{edge instances} of a template.
\begin{itemize}
  \item WordNet: \w{cat} $\rightarrow$ \w{feline} \textbf{vs.} \w{cup} $\rightarrow$ \w{container}.
  \item Nearest neighbors distance.
  \item Each \textit{edge instance} has a distance $f$.
\end{itemize}
\vspace{0.5cm}
\pause

\begin{tabular}{ll}
\hh{Cost of an edge is} & $\theta_i \cdot f_i$. \\
\pause
\hh{Cost of a path is} & $\theta \cdot \mathbf{f}$. \pause \\
\multicolumn{2}{l}{\hh{Can learn parameters $\theta$}.}
\end{tabular}

\end{frame}


%%%%%%%%%%%%%%%%%%%%%%%%%%%%%%%%%%%%%%%%%%%%%%%%%%%%%%%%%%%%%%%%%%%%%%%%%%%%%%%
% RESULTS
%%%%%%%%%%%%%%%%%%%%%%%%%%%%%%%%%%%%%%%%%%%%%%%%%%%%%%%%%%%%%%%%%%%%%%%%%%%%%%%

%%%%%%%%%%%%%%%%%%% 
% ConceptNet INTRO
%%%%%%%%%%%%%%%%%%%
\begin{frame}{Experiments}
\hh{ConceptNet:}
\begin{itemize}
  \item A semi-curated collection of common-sense facts. \\
    \vspace{0.1cm}
    \true{not all birds can fly} \\
    \true{noses are used to smell} \\
    \true{nobody wants to die} \\
    \true{music is used for pleasure}
    \vspace{0.1cm}
  \item Negatives: ReVerb extractions marked false by Turkers.
  \item Small (1378 train / 1080 test), but fairly broad coverage.
\end{itemize}
\vspace{0.5cm}
\pause

\hh{Our Knowledge Base:}
\begin{itemize}
  \item 270 million lemmatized Ollie extractions.
\end{itemize}
\end{frame}
  
%%%%%%%%%%%%%%%%%%% 
% ConceptNet RESULTS
%%%%%%%%%%%%%%%%%%%
\begin{frame}{ConceptNet Results}
\hh{Systems}
\begin{itemize}
  \item[] \textbf{Direct Lookup}: Lookup by lemmas.
  \item[] \textbf{Thesis}: This thesis.
  \pause
  \item[] \textbf{Thesis - Lookup}: Remove query facts from KB.
\end{itemize}
\pause

\begin{center}
  \begin{tabular}{lcc}
    System             & P     & R    \\
    \hline
    Direct Lookup      & 100.0 & \textbf<5-5>{12.1} \\
    \pause
    Thesis             & 90.6  & \textbf<5-5>{49.1} \\
    Thesis - Lookup    & 88.8  & 40.1 \\
  \end{tabular}
\end{center}
\pause

\begin{itemize}
  \item 4x improvement in recall.
\end{itemize}
\end{frame}

